\documentclass[a4paper]{article}
\usepackage{polski}
\usepackage[utf8]{inputenc}
\usepackage{graphicx}
\usepackage{hyperref}
\usepackage{float}
\usepackage{listings}

\title{Komputerowe wspomaganie projektowania inżynierskiego}
\author{Jakub Mandra}
\date{semestr letni 2016/2017}

\begin{document}

\begin{figure}
   \includegraphics[width=12cm]{utp_logo.jpg}
\end{figure}

\maketitle

\newpage

\section{Menu}

\begin{enumerate}
    \item Założenia projektu
    \item Prezentacja aplikacji
    \item UML dla aplikacji
    \item Kod źródłowy aplikacji
    \item Projekt piętra budynku
\end{enumerate}  

\newpage

\section{Założenia projektu}

{Projekt obejmuje aplikację w dowolnym języku/technologii do obliczania miesięcznego zużycia energii elektrycznej w domu, UML opisujący stworzoną aplikację, projekt jednego piętra dowolnego budynku wykonany w programie AutoCad, oraz dokumentację napisaną w LaTex.}

\section{Prezentacja aplikacji}
{Do stworzenia kalkulatora zużycia energii elektrycznej w domu użyłem technologii webowych.}
\newline{Aplikacja została napisana w oparciu o wzorzec projektowy Model-Widok-Kontroler (MVC).}
\newline{Modelem w mojej aplikacji są objekty przechowujące dane o urządzeniach napisane w JavaScript, widok zbudowany jest w HTML5 oraz CSS3 z zachowaniem responsywności i z wykorzystaniem prekompilatora Sass, a kontrolerem jest skrypt JS napisany przy użyciu AngularJS - frameworku JavaScript.}
\newline{Dodatkowo aplikacja bazuje na JavaSript`owym serwerze Node.js, który obsługuje rządanie przeglądarki do wczytania templateu i skryptów aplikacji.}
\\
\\
\href{http://power-consumption-app.herokuapp.com/}{http://power-consumption-app.herokuapp.com/}
\newline{Live preview działającej aplikacji.}
\\
\\
\href{https://github.com/Rotarepmi/Studies/tree/master/Power Consumption App}{https://github.com/Rotarepmi/Studies/tree/master/Power Consumption App}
\newline{Repozytorium GitHub.}

\begin{figure}[h]
   \includegraphics[width=12cm]{app_screen.jpg}
   \caption{Widok aplikacji}
\end{figure}

\newpage

{Struktura aplikacji:}

\begin{itemize}
    \item app.json
    \item package.json
    \item server.js
    \item index.html
    \item view
    \begin{itemize}
        \item main.html
        \item mockup.html
    \end{itemize}
    \item js
    \begin{itemize}
        \item angular.min.js
        \item angular-route.min.js
        \item jquery-3.2.1.min.js
        \item stickyCtrl.js
        \item main.js
    \end{itemize}
    \item css
    \begin{itemize}
        \item style.css
        \item style.css.map
    \end{itemize}
    \item sass
    \begin{itemize}
        \item style.sass
    \end{itemize}
    \item node-modules
\end{itemize}

\newpage

\section{UML dla aplikacji}

\begin{figure}[h]
   \includegraphics[width=10cm]{uml_usecase.jpg}
\end{figure}

\begin{figure}[h]
   \includegraphics[width=10cm]{uml_classes.jpg}
\end{figure}

\newpage

\section{Kod źródłowy aplikacji}


\lstinputlisting[basicstyle=\tiny, tabsize=1, caption=server.js]{server.js}
{Skrypt serwera jest prosty, obsługuje tylko wczytywanie głównego indexu strony. W razie niepowodzenia wysyła do przeglądarki użytkownika błąd o numerze 404 (nie znaleziono pliku).}
     
\lstinputlisting[basicstyle=\tiny, tabsize=1, caption=package.json]{package.json}
{Jest to lista zależności potrzebnych podczas instalowania serwera Node.js. Określamy wersję aplikacji oraz elementy, które mają zostać zainstalowane, repozytorium aplikacji i ścieżkę do skryptu absługującego server.}
     
\newpage 
 
\lstinputlisting[basicstyle=\tiny, breaklines= true, tabsize=1, caption=index.html]{index.html}
{Główny plik HTML, który jest wczytywany do przeglądarki. Jest to niejako model aplikacji. Pozycja startowa, gdzie dopiero później wstrzykiwane są kolejne widoki i skrypty aplikacji.}

\lstinputlisting[basicstyle=\tiny, breaklines= true, tabsize=1, caption=view/main.html]{main.html}
{Template strony napisany w HTML, zawiera rozwijaną listę i kontenery wyświetlające informacje. Wszsytko obsługiwane przez AngularJS.}

\lstinputlisting[basicstyle=\tiny, breaklines= true, tabsize=1, caption=view/mockup.html]{mockup.html}
{Template kafelki prezentującej dodane do zestawienia urządzenie. Do inputów podpięte są funkcje Angulara, cały widok aktualizowany jest dynamicznie przy każdej zmianie parametrów urządzeń.}

\lstinputlisting[basicstyle=\tiny, breaklines= true, tabsize=1, caption=sass/style.sass]{style.sass}
{Style strony napisane w prekompilatorze Sass. Opisują wygląd każdego elementu aplikacji}

\lstinputlisting[basicstyle=\tiny, breaklines= true, tabsize=1, caption=css/style.css]{style.css}
{Przekompilowane na css style aplikacji, które mogą być odczytywane i interpretowane przez przeglądarkę}

\lstinputlisting[basicstyle=\tiny, breaklines= true, tabsize=1, caption=js/main.js]{main.js}
{Główny plik js, w którym zawarty jest skrypt aplikacji. \\ \\
Utworzony jest module Angulara, z potpiętym ngRoute - pozwala na wczytywanie widoków strony bez przeładowania. \\ \\
routerProvider odpowiedizalny jest za wczytanie templateu strony zawartego pod podaną ścieżką. \\ \\
dyrektywa addMockup wstrzykuje template do kontenera. \\ \\
appCtrl - główny kontroler aplikacji, działa na elementach w scopie (czyli elemenatach w kontenerze, do którego jest podpięty kontroler), zawiera: \\ \\

\$scope.kitchenDevs, bathroomDevs, roomDevs i gardenDevs - są to listy obiektów, które reprezentują posegregowane urządzenia, które można dodać do zestawienia kalkulatora (pierwotnie planowałem umieścić je w osobnym pliku .json, jednak nie pozwoliłoby to na przechowywanie wewnątrz obiektów funkcji); \\ \\

\$scope.hideList - funkcja zmienia property display kontenera listy, wywoływana jest po opuszczeniu pola listy przez kursor, zdarzenie obsługiwane jest przez Angular; \\ \\

\$scope.toggleList - jak wyżej, z tymże zdarzeniem jest ng-click; \\ \\

\$scope.addMockup - w widoku wyświetlana jest dynamicznie aktualizowana lista z kafelkami urządzeń, wybranie urządzenia z listy wywołuje niniejszą funkcję, która dodaje template do listy i aktualizuje widok; \\ \\

\$scope.deleteMockup - usuwa z listy template danego urządzenia i aktualizuje widok; \\ \\

\$scope.changeOfValue - funkcja wywoływana za każdym razem, gdy użytkownik zmieni wartość w inpucie urządzenia; \\
aktualizuje początkowe wartości wylicozne przez funkcje zawarte w obiekcie urządzenia i aktualizuje widok; \\ \\

\$scope.updatePrice - funkcja wywołana tylko w przypadku zmiany ceny za energię ele., aktualizuje widok; \\ \\

funckja updateValues() - funckja anonimowa wywoływana zawsze aby zaktualizować przekalkulowane wartośći zużycia prądu i opłat, zbiera wartości zmeinnych z całej listy urządzeń będących dodanych do widoku;
}

\newpage 

\section{Projekt piętra budynku}

\begin{figure}[h]
   \includegraphics[width=12cm]{wymiarowanie.jpg}
   \caption{Schemat budynku z wymiarowaniem}
\end{figure}

\newpage

\begin{figure}[h]
   \includegraphics[width=12cm]{instalacja.jpg}
   \caption{Schamat budynku z instalacją elektryczną}
\end{figure}

\end{document}
